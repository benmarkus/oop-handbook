\subsection{Az osztály}
\noindent

Az adattípus fogalmának megismerése után ismerkedjünk meg az osztály fogalmával.

\bigskip
Az objektumelvű tervezés során osztályként adjuk meg az általunk definiált egyedi típusokat (custom type). Az osztály egy objektum szerkezetének és viselkedésének mintáját adja meg. Felsorolja az objektum \code{adattagjainak} nevét, típusát, láthatóságát, opcionálisan megadhatja az egyes adattagok típusinvariánsait, valamint megadja az objektumra meghív\-ható \code{metódusokat} névvel, paraméterekkel, visszatérési típussal, metódustesttel és látható\-ság\-gal ellátva.

\paragraph{Láthatóság:}
Az egyes adattagok és metódusok láthatósága azt mondja meg, hogy az adott objektum adattagját és metódusát milyen helyekről tudjuk elérni. A láthatóság megvalósítása eltér az objektumelvű nyelvek között.

\bigskip
Pythonból (és a legtöbb scriptnyelvből) ez a nyelvi elem például hiányzik, ezt konvenció szerint alsóvonásokkal pótolhatják a fejlesztők, azonban ezt a Python interpreter nem tartatja be szigorúan, kizárólag a fejlesztők egymás közötti kommunikációját szolgálja. Ezzel szemben Java-ban 4 láthatósági beállítás van (access modifier), C\#-ban 6-ot különböztetünk meg, ezek közül 3-at fogunk használni a félév során:

\paragraph{Public:}
Bárhonnan elérhető adattag és metódus. Jele: +.
\paragraph{Private:}
Csak osztályon belül elérhető adattag és metódus. Jele: -.
\paragraph{Protected:}
Csak osztályon belül, és leszármazottain (lásd: Öröklődés) belül elérhető adattag és metódus. Jele: \#.