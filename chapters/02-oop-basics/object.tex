\subsection{Az objektum}
\noindent

Az Objektumelvű programozás központi eleme az objektum. Objektumnak a feladat azt az önálló egyedként azonosított részét nevezzük, amely magába foglalja az általa ellátott részfeladat megoldásáért felelős adatokat és műveleteket. Az objektumokra létrehozásuk után közvetett módon, egy változóval tudunk hivatkozni. Ez a változó tartalmazza a program futása közben az objektum által használt memóriaterület címét.

\begin{csharpblock}
Object obj1 = new Object();
\end{csharpblock}

A fenti kódrészletben a \code{new} kulcsszót használva hozzuk létre az objektumot, az \code{Object()} adja meg hogy milyen objektumot hozunk létre (jelen esetben ez az \code{Object} ősosztály), az \code{obj1} pedig a fent említett változó.

